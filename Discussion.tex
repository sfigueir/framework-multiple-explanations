

\section{Discussion}\label{sec:Discussion}


 In this work,  we  design  an experimental framework in which participants observe an incomplete set of examples, which are consistent with two alternative minimal descriptions depending on which features are observed.  We  illustrate  several advantages of our method compared to separately presenting sets of examples consistent with only one minimal description at a time. First, we  show  that when a set of examples is consistent with a disjunction \textit{and also} with a conjunction, participants are more likely to find the conjunction, in accordance with well-known previous results that show that the conjunction is learnt faster than the disjunction when presented separately \cite{bourne1970knowing}. Then, we  show  that when rules of significantly different MDL are consistent with the observations, almost all participants discover the simpler rules, consistent with previous result showing that, when rules of different MDL are tested separately, learning times are proportional to MDLs \cite{feldman2000minimization}. Finally, we  show  that when the logical structure of the minimal rules is independent of the selected features, participants are more likely to reuse the same features used to describe previous concepts,  and preliminary results suggest that reusing features  allows them to learn concepts faster than a control group that is not reusing features. To our knowledge this effect has not been previously characterized in the concept-learning literature, adding to the library of effects illustrating how human attention is biased towards features that are useful to describe the concepts (see \cite{blair2009extremely,kruschke2000blocking,kruschke2005eye,hoffman2010costs}, among others).


Eye-tracking studies in categorization tasks have revealed that feature attention rapidly changes between trials depending on which features are relevant for classification in each trial~\cite{blair2009extremely}, as well as depending on prior knowledge about feature relevance \cite{kim2011prior}. In \cite{kruschke2005eye} it is found that eye movements confirmed that attention was learned in the basic learned inhibition paradigm, and in \cite{hoffman2010costs} it is also found that eye movements revealed how an attention profile learned during a first phase of learning affected a second phase.  Our experimental setup allows us to test an arguably simpler complementary hypothesis: everything else being equal, participants are biased to use the same features used in the past.  Importantly, we were only able to test this hypothesis thanks to our framework, which allows us to present a set of examples consistent with two rules of exactly the same logical structure, but using different sets of features. Then, without using eye-tracking, we can recover which rule the participants learned, and thus which set of features they attended to. Since the two sets of features explain the examples using exactly the same logical structure, preferentially explaining the concept using one set of features over the other can only be due to a preference over the features themselves, and not a preference over alternative logical structures. 





Although some of the hypothesis that we test are aligned with the well-known Einstellung effect which states that adopted solutions may hinder simpler ones when aiming at tackling novel problems, our experimental setting is different to the classical water jar test (the most commonly cited example of an Einstellung effect, where participants need to discover how to measure a certain amount of water using three jars with different and fixed capacity)  \cite{luchins1942mechanization}
in two senses. First, we do not drive the experiment to control and supervise the aspects that participants have to pay attention to. On the contrary, our focus is on the {\em choice} of the features that show to be useful for learning a concept with more than one rational explanation. Second, our experimental framework is consistent with the Language of Thought (LoT) hypothesis \cite{fodor1975language}, which states that the human capacity to describe concepts —and, more generally, of all elements of thought— builds on the use of a symbolic and combinatorial mental language
and it is specifically conceived to handle expressions in propositional Logic (but expansible to other formal languages), which is the ground where the rational explanations can be formalized. Such approach enables us to treat the notion of {\em feature} in a very precise way. 

We note that other frameworks besides LoT can be used for our experiment. For example, consider similarity-based classification rules \cite{juslin2003cue,juslin2003exemplar}, where each feature is multiplied by a weight and the classification rule is a function of the sum of the weighted features, usually a linear function with a soft decision boundary \cite{juslin2003exemplar}. In this framework, the generalization phase would determine which of two possible decision boundaries was used by the participants (both consistent with the elements observed in the learning phase); and the feature-stickiness effect would be explained by the inertia of the weights' values from one concept to the next. However, two obstacles in this framework makes us prefer the LoT framework for Boolean concept-learning tasks. First, although a linear classification rule can readily learn the conjunctions and disjunctions in our experiment, more complex classification rules would require nonlinear functions of the features (e.g.\ the exclusive-or (XOR)). For nonlinear boundaries, the values of the weights that accompany the features could be hard to interpret, since it might no longer be true that a higher weight means higher feature importance. In contrast, in the LoT framework complex classification rules are compositionally built to accommodate concepts of any complexity, and feature importance can always be modeled as the probability of including a feature in a formula, independently of its complexity. Second, unlike similarity-based rules, the LoT framework naturally explains how humans can built verbal explanations for the learned concepts. Indeed, almost all participants gave informal explanations of conjunctions and disjunctions in propositional logic after learning each concept (see the shared data online for the list of verbal explanations). 



Another well-studied phenomenon related to our work is Kamin’s cue {\em blocking}, where the learning of a given stimulus B is {\em blocked} by the mere fact that it was preceded by a set of stimuli A that already pairs with the outcome. This shows that the subject learned that stimulus B was not useful, and hence disregards their attention to it in the upcoming events \cite{wagner1970stimulus,mackintosh1975theory,rescorlaw72}. Studied in humans in \cite{chapman1990cue, arcediano1997blocking, kruschke2000blocking} among others, our work differs from these approaches in that we never introduce a stage were a feature A is intentionally exposed in absence to B, in order to guide the attention of the participant.



We conjecture that most first-order determinants of subjective concept difficulty will also hold in a relative manner in our dual-concept setup, such as the MDL bias (for less extreme cases than evaluated in this work) \cite{feldman2003simplicity} and the transfer learning hierarchical structure bias \cite{tano2020towards}. Importantly, our experimental setup also allows to directly test second-order subjective difficulty effects (e.g.\ concept A is learnt faster if presented jointly with concept B than with concept C), as well as second-order transfer learning effects (e.g.\  participants learn more rapidly concept C if they have first observed concept A coupled with B$_1$, compared to A coupled with B$_2$). We believe that a systematic study of concept-learning difficulty with two (or more) concepts presented at the same time in each trial may open a new window into the dynamics of human concept-learning mechanisms. For example, consider the study in \cite{piantadosi2016logical}, where participants gradually learn one concept while simultaneously selecting elements currently believed to belong to that concept. Here, the authors fit a Bayesian language model to participants' choices in order to illustrate how the posterior probability of the different rules in the grammar varied across time, to approximate the order in which different rules are learned. In contrast, using our experimental setting we can directly estimate, in a model-free manner, the probability that each rule is learnt faster than another. One simply needs to jointly present (in an incomplete and mutually compatible way) a set of examples consistent with those two minimal rules, and then measure the fraction of participants that discover each rule.

Usually, concept-learning biases have been studied in an isolated manner: the participant observes examples indicated as inside or outside a \textit{single} concept, and the experimenter evaluates its subjective difficulty for the participant. Although different methods have been used to present the concept to the participant (e.g.\ all elements at the same time \cite{tano2020towards,kemp2012exploring} or small sets of elements presented in series \cite{piantadosi2016logical}), to the best of our knowledge all previous category-learning studies have attempted to evaluate a single concept at a time. Here, we present a controlled logical setting to evaluate the relative difficulty of two concepts presented at the same time and under the same experimental conditions, and the framework could be generalized to more concepts straightforwardly. 
